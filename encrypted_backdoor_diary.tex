% =============================================================================
% Research Diary Usage Example
% =============================================================================
% 
% This document demonstrates how to use the research diary system
% with detailed examples for each environment and feature.
%
% Author: PhotonZhang
% Email: zyw23@mails.tsinghua.edu.cn
% Collaborator: Claude Sonnet 4 (AI Assistant)
% Version: 1.0
% Release Date: July 29, 2025 (Beijing Time)
% Date: 2025
% =============================================================================

\documentclass[12pt,a4paper,twoside]{article}

% Load the research diary style package
\usepackage{researchdiary}

% Add bibliography file
\addbibliographyfile{encrypted_backdoors_ref.bib}

% =============================================================================
% DOCUMENT BEGINS
% =============================================================================

\begin{document}

% Create title page
\maketitlepage{Annabel Jakob}{Improving Obfuscation and Robustness of Encrypted Backdoors}

% Add table of contents
\newpage
\tableofcontents
\newpage

\section{13th February 2026}

\begin{meetingsummary}
    \paragraph{Uploading the paper to arxiv}
    \begin{itemize}
        \item Agreed to put paper on arxiv
        \item Work to be done before uploading:
        \begin{itemize}
            \item Add authors
            \item Add acknowledgements
            \item Remove ICML references
        \end{itemize}
    \end{itemize}

    \paragraph{Extensions for rebuttal}
    \begin{itemize}
        \item Symmetry transformations
        \item Matrix transformations
        \item End-to-end model
    \end{itemize}

    \paragraph{Symmetry transformations}
    \begin{itemize}
        \item Andis' symmetry transformations introduce pairs of transformations that cancel each other out
        \item[$\rightarrow$] E.g. sample a random rotation matrix
        \item[$\rightarrow$] Can only apply permutations to the $\mathbf{w}_\text{SiLU}$
    \end{itemize}

    \paragraph{End-to-end model}
    \begin{itemize}
        \item Current issues with end-to-end model:
        \begin{itemize}
            \item memory management
            \item skip connections
            \item[$\rightarrow$] This is mainly an engineering issue
        \end{itemize}
        \item Alternative: have the entire construct in a single layer (see Gemma notebook)
        \item The two main tasks for creating a backdoor in full Transformer models:
        \begin{enumerate}
            \item Make it possible to reuse features. I.e. prevent the skip connection from corrupting the output of each layer.
            \item Implement aggregation of features from many input positions via the attention mechanism. E.g. we map each input token to a 0/1 bit, then aggregate these bits to one position where we run the backdoor circuit. The problem to avoid is layer norm scaling each bit by a different factor, making it unsuitable for the backdoor circuit.
        \end{enumerate}
    \end{itemize}

    \paragraph{Robustness}
    \begin{itemize}
        \item In my thesis, I only measured how much noise the backdoored models can withstand, but they actually only need to withstand as much noise as the target model can withstand
        \item[$\rightarrow$] Task: measure how robust normal transformers are (see original backdoor paper~\cite{draguns2025unelicitable} for details on how to do this)
        \item Formal definition for measuring distance to base model?
    \end{itemize}
    
\end{meetingsummary}

% Print bibliography
\printbibliographysection

\end{document} 